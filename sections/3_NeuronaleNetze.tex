\section{Neuronale Netze}
Ein Neuron ist ein Schwingkreis. Es hat n Eingänge und einen Ausgang. Die Werte der
Eingänge werden summiert. Wenn die Summe der Eingänge einen Schwellwert übertrifft, wechselt der Ausgang, das Neuron feuert.
Für die Schwellwertfunktion, wird entweder die Signum (0/1) oder die Sigmoid-Funktion welche ``weiche'' Kanten hat verwendet.
\subsection{Klassisches Neuron / Perzeptron}
Das Perzeptron ist das klassische mathematische Neuron. Es besteht aus:
\begin{itemize}
	\item Summierer
	\item Feuerschwelle
	\item Ausgabefunktion (Sigmoid)
\end{itemize}
\subsection{Backpropagation}
\subsection{Hopfield}
\subsection{Kohonennetze}

